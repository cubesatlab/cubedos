\chapter{Tutorial}
\label{chapt-tutorial}

Merc is a compiler that takes the definition of a CubedOS module interface, written in a
suitable interface definition language (IDL), and outputs an API package for that module in
SPARK. The intention is for the generated code to be provably free from runtime errors without
any further intervention on the part of the developer. Thus Merc greatly simplifies the work
involved in creating API packages for CubedOS modules.

The IDL used as input to Merc is an extension of the XDR language described in RFC-4506. The
extended features allow expression of constraints on scalar types, as well as some additional
features pertaining to the SPARK target language. We call this extended version of XDR
``modified XDR'' or MXDR. The central abstraction in MXDR is that of the message. Rather than
declaring remote procedures directly, or specifying classes with methods as is done in most
OOP-based IDLs, MXDR allows you to describe messages by the information they contain. Each
message implies three subprograms:

\begin{enumerate}
\item A function that takes the message contents as typed \texttt{in} parameters and encodes the
  message into a raw octet array for transmission.
\item A procedure that decodes the message as a raw octet array back into typed \texttt{out}
  parameters.
\item A function that verifies that a given octet array contains a instance of a particular
  message.
\end{enumerate}

The MXDR user only need to specify the structure of the message. Merc generates the three
associated subprograms without any additional assistance. This message-centric focus is how MXDR
differs from most interface definition languages, and is one of the main reasons we developed an
entirely new tool for CubedOS.

We choose to base Merc on XDR to simplify interoperability with other systems in the future. At
the time of this writing, CubedOS only supports messages between modules in the same operating
system process, all of which are written in SPARK. However, we anticipate extending the
architecture to support message passing between processes, and even between machines. Using a
standard wire protocol, such as XDR, may simplify that work. All of the extensions to XDR that
we define are at the IDL source level only, and are used to guide the generation of provable
SPARK. We specify no changes to the underlying XDR format.

This section of the Merc documentation is a tutorial for Merc users. Here we describe how to set
up Merc, writing MXDR files, and then use Merc to generate API packages. When you are
contemplating a new CubedOS module, you should first express the API to your module as a
collection of messages it sends and received. Write an MXDR file that declares the structure of
those messages, and then use Merc to generate the API package. You can then focus the bulk of
your programming effort on the part of the module that implements the messages you defined
without wasting time writing tedious and highly repetative message encoding and decoding
subprograms.

