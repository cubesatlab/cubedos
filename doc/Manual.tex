
\chapter{Manual}
\label{chapt:manual}

This chapter contains the user's manual for CubedOS. The intended audience of this chapter is
CubedOS application developers. Contributors to the CubedOS core and runtime library may also
find the information in this chapter useful.

\section{Module Organization}
\label{sec:module-organization}

In this section we describe the recommended structure of a basic CubedOS module. Modules are not
required to follow this structure exactly; special needs often require special handling.
However, the structure we describe here is appropriate for many modules. The CubedOS source
distribution includes a template module, with comments, that you can use as a starting point for
your own modules.

A module consists of a hierarchy of packages formed by a root package and (at least) two child
packages. The root package can be used to hold module-wide declarations as appropriate. In many
cases the root package may not require a body, although module-wide subprograms might usefully
be declared in the root package.

For a module named |My_Module| we recommend the following two or three child packages:

\begin{itemize}
\item |My_Module.Messages| is a public child package containing the server task that loops
infinitely reading the module mailbox and processing the messages it finds there. This loop
drives the activity of the module, processing exactly one message at a time. No other task in
the module, or indeed in the entire system, should read from the module's mailbox. Furthermore,
the server task should receive a message in only one place in the program. This highly stylized
structure is conceptually simple and easy to analyze.

\item |My_Module.Internals| is a private child package that contains the bulk of the module's
processing logic. In simple cases, this package may not be necessary and all processing of
messages can be in the |Messages| package. However, complex modules may find it useful to break
out the bulk of the processing logic into |Internals| to separate that logic from the CubedOS
infrastructure associated with message handling. The code in this package is not accessible from
outside the module (since it is a private child package), and it can assume it is called by only
one task unless additional, internal tasks are created by the module developer.

In even more complex cases, it may make sense for there to be multiple private "internals"
packages, or even a private internal package hierarchy. Organizing the internal logic of a
module is ultimately at the discretion of the module developer.

\item |My_Module.API| is a public child package that contains helper subprograms for
constructing and parsing module messages. These helper functions only manipulate message
records. They do no actual message processing. Other modules will call these subprograms to
build messages sent to and parse messages received from |My_Module|. The |API| package may also
include type definitions related to the module's public interface.

A separate tool, Mercury, can be used to autogenerate the |API| package from an interface
definition. Mercury is described in Chapter~\ref{chapt:mercury}.
\end{itemize}

Although this structure seems complex it has several advantages.

The bulk of the module's logic is in packages (both |Internals| and |API|) that, in most cases,
have no tasking structures. Typically, for modules with no internal tasks, all the code in the
|Internals| package is executed by only one task, simplifying development. The |API| encoding
and decoding subprograms might be called by multiple tasks, but they are relatively simple, with
no global effects, and thus easy to make task-safe.

Because the |Internals| package is a private child package, its resources cannot be directly
accessed by the rest of the program.

Using the module API is highly stylized and entails calling an API function to build a message
that is then sent to the module's mailbox. This is most conveniently done using |Route_Message|
in the |Message_Manager| package:

\begin{lstlisting}
   Route_Message(My_Module.API.Operation_Request(1, 2, 3));
\end{lstlisting}

Here |Operation_Request| is a function that creates a message causing |My_Module| to perform
some operation. The name of the function should be reflective of the operation being done. We
recommend using the convention of adding ``Request'' as a suffix to the names of messages making
requests sent to the module, and adding ``Reply'' as a suffix to the names of messages that are
replies from the module. This aligns with the concept that modules are, in effect, servers that
accept requests and return replies.

The arguments to the function (1, 2, and 3 in this case) are arguments to the operation and are
packed into a message following XDR rules. The Mercury tool auto-generates |API| bodies that
take care of this. Note that |Route_Message| handles looking up a module's ID in the name
resolver package. In the case of a multi-domain application, |Route_Message| will automatically
direct the message to some underlying message router, as needed.

Finally, the architecture of modules is compatible with the Ravenscar or Jorvik restrictions. In
particular, a module's server task is a single, library level task that runs infinitely, and all
communication with a module is by way of a protected mailbox object with only one entry for
receiving messages.

\section{Core Module Documentation}
\label{sec:core-module-documentation}

This section describes the core modules that are part of the CubedOS distribution. These modules
are intended to be used by most CubedOS applications. They provide basic services. Additional
information about a core module can be found in the comments in the module's top level package
and |API| package specifications, or in the MXDR file used by Mercury to generate the modules
|API| package.

\subsection{Log Server}
\label{sec:log-server}

The log server is a core module that provides a simple logging facility. The default log server
outputs log messages to the console via Ada's standard library package |Ada.Text_IO|. For
systems where that package is not available, developers can write their own body for package
|CubedOS.Log_Server.Messages| that handles incoming messages in whatever way makes sense for
their application.

Log messages are intended to be short strings of human-readable text. Each message has an
associated log level, described below, and is automatically timestamped by the log server with a
boot-relative time in seconds.

The log server's API package provides a convenience procedure named |Log_Message| that takes
care of the details of constructing a message and sending it to the log server. It is expected
that most uses of the Log Server will be by way of calling |Log_Message|. Currently, the Log
Server does not send any reply messages. Logging is assumed to always be successful. This is an
area for future improvement.

Of particular interest is the |Log_Level| parameter to |Log_Message|. This parameter is used to
indicate the severity of the log message. It is an enumeration type. The Log levels are intended
to follow those used by the Unix syslog facility, both in name and meaning. The following log
levels are defined:

\begin{itemize}
\item |Debug|: Messages that are only of interest to developers. These messages are typically
very detailed and are used to diagnose problems in the system. Debug messages are not typically
enabled in a production system.

\item |Informational|: Messages that are of interest to users. These messages are typically
high-level and are used to inform the user of the system's normal operation.

\item |Notice|: Like informational messages, notice messages occur during normal system
operation. However, they log particularly significant events that are meant to stand out
compared to ordinary informational messages.

\item |Warning|: Messages that indicate a potential problem. The system is still functioning
correctly, but the message indicates that something might be wrong.

\item |Error|: Messages that indicate that a problem has occurred. The system is still
functioning, but the message indicates that something unintended or unexpected has happened.

\item |Critical|: Critical messages are similar to error messages, except with more severity and
criticality. A critical message indicates that the system is in a bad state and that the problem
should be addressed soon if the system is to continue functioning normally.

\item |Alert|: Messages that indicate a problem that has occurred that is so severe that the
system cannot function normally without corrective action. The system will likely soon fail
completely.

\item |Emergency|: Emergency messages are the most severe. They indicate that the system has
failed, is unusable, or soon will be, and that immediate action is required.
\end{itemize}

The timestamps added to every log message show the time the message was processed by the log
server relative to when the system started. Timestamps are in seconds. This time includes the
overhead of passing the message from the sending module to the log server. Because multiple
messages might be queued in the log server's mailbox, and because of task priority issues, the
timestamp might indicate a time that is, potentially, significantly later than when the message
was generated by the sending module.

One area for future work in the log server would be for the sending module to generate a
timestamp immediately before when the message is sent and pass that timestamp to the log server
as part of the message. In that case, the timestamps would be a more accurate reflection of when
the message was generated, being free from queuing and task priority effects. The convenience
procedure |Log_Message| could take care of the timestamp management details, so the developer
wouldn't need to do any extra work.

Of course, many applications won't care about such precise timestamps. The current
implementation will likely be sufficient for those applications. However, applications with very
tight timing requirements might benefit from better debugging and monitoring that would be made
possible by more accurate timestamps.

An implementation that displayed timestamps using an absolute time was considered. However, such
an implementation would need access to a real-time clock via, for example, package
|Ada.Calendar|. While |Ada.Calendar| is supported under the Jorvik profile, some small embedded
environments nevertheless do not support it. To keep CubedOS as widely applicable as possible,
it was decided to avoid the use of |Ada.Calendar| and instead focus on only boot-relative
timestamps.

Other future enhancements to the log server include (but are not limited to) the following:
\begin{itemize}
\item There should be a way to specify the log level that is displayed. Under normal operation,
users will not want to see Debug messages, for example. Furthermore, because such messages might
be plentiful and could potentially consume significant resources to transmit in volume, the
filtering of those messages should be done in |Log_Message| and not by the log server itself.
That would avoid transmitting messages that are of no interest.

\item As an extension of the idea above, it should be possible to control the log level based on
the module ID of the sending module. The use-case for this ability is when the developer would
like to log Debug messages from a particular module being debugged without having to see all
Debug messages generated from all other modules too.
\end{itemize}