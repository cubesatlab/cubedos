
\chapter{Manual}
\label{chapt:manual}

This chapter contains the user's manual for CubedOS. The intended audience of this chapter is
CubedOS application developers. Contributors to the CubedOS core and runtime library may also
find the information in this chapter useful.

\section{Module Organization}
\label{sec:module-organization}

In this section we describe the recommended structure of a basic CubedOS module. Modules are not
required to follow this structure exactly; special needs often require special handling.
However, the structure we describe here is appropriate for many modules. The CubedOS source
distribution includes a template module, with comments, that you can use as a starting point for
your own modules.

A module consists of a hierarchy of packages formed by a root package and (at least) three child
packages. The root package can be used to hold module-wide declarations as appropriate. The most
important such declaration is the module's ID number. However, module-wide type declarations, or
API elements that are shared with the internals package (described below) would also be
candidates for inclusion in the root package. In many cases the root package may not require a
body, although module-wide subprograms might usefully be declared in the root package.

For a module named |My_Module| we recommend the following three child packages:

\begin{itemize}
\item |My_Module.Messages| is a public child package containing the server task that loops
  infinitely reading the module mailbox and processing the messages it finds there. This loop
  drives the activity of the module, processing exactly one message at a time. No other task in
  the module, or indeed in the entire system, should read from the module mailbox. Furthermore
  the server task should receive a message in only one place in the program. This highly
  stylized structure is conceptually simple and easy to analyze.

\item |My_Module.Internals| is a private child package that contains the bulk of the module's
  processing logic. The message loop is intended to be minimal; it calls into the |Internals|
  package as soon as it can. The code in this package is not accessible from outside the module
  and it can assume it is called by only one task (unless additional, internal tasks are created
  by the module developer).

\item |My_Module.API| is a public child package that contains helper functions for constructing
  and parsing module-specific messages. These helper functions only manipulate message records.
  They do no actual message processing. Consequently they can be easily written as pure
  functions and made task-safe. Other modules will call these functions to build messages sent
  to and parse messages received from |My_Module|. The |API| package may also include type
  definitions related to the module's public interface.
\end{itemize}

Although this structure seems complex it has several advantageous features.

First the bulk of the module's logic is in packages (both |Internals| and |API|) that have no
tasking structures. At the time of this writing \SPARK\ does not yet support Ada tasking and so
this allows |SPARK_Mode| to be on for the two packages that do most of the processing while
remaining off for the relatively trivial package containing the message loop.

Because the |Internals| package is a private child package, its resources cannot be directly
accessed by the rest of the program. This ensures that no tasks unknown by the module author
will be calling internal subprograms.

Using the module API is highly stylized and entails calling an API function to build a message
that is then sent to the module's mailbox. For example:

\begin{lstlisting}
   Mailboxes(My_Module.ID).Send(My_Module.API.Operation_Message(1, 2, 3));
\end{lstlisting}

Here |Operation_Message| creates a message causing |My_Module| to perform some operation. The
name of the function should be reflective of the operation being done. Here ``operation'' is
used as a place holder for some more descriptive name. We recommend using the convention of
adding ``Message'' as a suffix to the names of all message constructing functions.

The arguments to the function (1, 2, and 3 in this case) are arguments to the operation and are
packed into a message using some module-specific means; module authors have total control over
the formatting of their messages so the message format need not be made public information.
\pet{Currently the message record type is not private, but it may be possible to use type
  extension to fix this (each module provides a private extension). That would reduce potential
  problems by prohibiting module users from editing the contents of messages.} The resulting
message returned by |Operation_Message| is then sent to |My_Module|'s mailbox.

Finally the architecture of modules is compatible with the Ravenscar restrictions. In
particular: a module's server task is a single, library level task that runs infinitely, and all
communication with a module is by way of a protected mailbox object with only one entry for
receiving messages.
