
\section{Overview}
\label{sec:overview}

WARNING: This section is outdated as CubedOS now uses the Jorvik profile. It is still here because it is helpful in understanding the original design considerations leading to the present architecture.

To understand the context of the CubedOS architecture, it is useful to compare the architecture
of a CubedOS-based application with that of a more traditional application. CubedOS abides
by the restrictions Jorvik Ada tasking profile, but it is helpful to compare its structure with ravenscar based applications.

Figure~\ref{fig:traditional-architecture} shows an example application using Ravenscar tasking.
Tasks, which must all be library level infinite loops, are shown in open circles and labeled as
$T_1$ through $T_4$. Tasks communicate with each other via protected objects, shown as solid
circles and labeled as $PO_1$ through $PO_4$.

\begin{figure}[tbhp]
  \center
  \scalebox{0.50}{\includegraphics*{Figure-Traditional.pdf}}
  \caption{Traditional Ravenscar-based Architecture}
  \label{fig:traditional-architecture}
\end{figure}

Arrows from a sending task to a protected object indicate calls to a protected procedure to
install information into the protected object. Arrows from a protected object to a receiving
task indicate calls to an entry in the protected object used to pick up information previously
stored in the object. Entry calls will block if no information is yet available but protected
procedure calls do not block.

Ravenscar requires that protected objects have at most one entry and that at most one task can
be queued on that entry. In CubedOS applications each protected object is serviced by exactly
one task. This ensures that two tasks will never accidentally be queued on the protected
object's entry \emph{provided the mapping from task (actually module, see below) to protected
  object is truly bijective}. In the figure this means only one arrow can emanate from a
protected object, and each task can have only one arrow incident upon it. Note that this
architecture is actually more restrictive than what Ravenscar allows. However, in any case,
multiple arrows can lead to a protected object, since it is permitted in CubedOS for many tasks
to call the same protected procedure or for there to be multiple protected procedures in a given
protected object.

In the example application of Figure~\ref{fig:traditional-architecture}, tasks $T_1$ and $T_3$
call protected procedures in two different protected objects. This presents no problems since
protected procedures never block, allowing a task to call both procedures in a timely manner.
However, task $T_3$ calls two entries, as allowed by Ravenscar but not CubedOS, one in $PO_1$
and another in $PO_2$. Since entry calls can block, this means the task might get suspended on
one of the calls leaving the other protected object without service for an extended time. The
application needs to either be written so that will never happen or be such that it doesn't
matter if it does.

There are several advantages to the traditional organization:

\begin{itemize}
\item The protected objects can be tuned to transmit only the information needed so the overhead
  can be kept minimal.
\item The parameters of the protected procedures and entries specify the precise types of the
  data transferred so compile-time type safety is provided.
\item The communication patterns of the application are known statically, facilitating analysis.
\end{itemize}

However the traditional architecture also includes some disadvantages:

\begin{itemize}
\item The protected objects must all be custom designed and individually implemented, creating a
  burden for the application developer.
\item The communication patterns are relatively inflexible. Changing them requires overhauling
  the application.
\item Third-party library components are awkward to write since they would have to know about
  the protected objects that are available in the application in order to communicate with other
  application-specific components. This is a particular problem for client/server oriented
  systems where a general purpose, reusable server component needs some way to send replies to
  clients for which it has no prior knowledge. \footnote{With access types, the client could
    potentially send a ``return address'' consisting of an access value that points at a
    client-owned protected object implementing an agreed upon synchronized interface. This design
    is ruled out by \SPARK.}
\end{itemize}

The CubedOS version of the sample application has an architecture as shown in
Figure~\ref{fig:cubedos-architecture}. In this case CubedOS provides the communication
infrastructure as an array of general purpose, protected mailbox objects. CubedOS
\textit{modules} communicate by sending messages to the receiver module's mailbox. The messages
are unstructured octet streams, and thus completely generic. Each module has exactly one mailbox
associated with it and contains a single task (the \newterm{server task}) dedicated to servicing
that mailbox alone, creating a one-to-one relationship between modules and mailboxes. A module's
mailbox servicing task extracts messages from the mailbox, decodes the messages, and then acts
on the messages, perhaps sending messages to other modules in the process.

Note that CubedOS modules are allowed to have additional internal tasks, if required, as long as
the usual Ravenscar rules are obeyed. These internal tasks can send messages to other modules,
but they do not attempt to receive messages; that work is reserved for the server task alone.

The communication connections shown in Figure~\ref{fig:cubedos-architecture} are the same as
those shown in Figure~\ref{fig:traditional-architecture} except that the two communication paths
from $T_1$ to $T_3$ are combined into a single path going through one mailbox.

\begin{figure}[tbhp]
  \center
  \scalebox{0.40}{\includegraphics*{Figure-CubedOS.pdf}}
  \caption{CubedOS-based Architecture}
  \label{fig:cubedos-architecture}
\end{figure}

CubedOS relieves the application developer of the problem of creating the communications
infrastructure manually. Adding new message types is simplified with the help of a tool,
\texttt{Merc}, stored in a different repository of the CubeSat Laboratory GitHub. In addition to
providing basic, bounded mailboxes, CubedOS also provides other services such as message
priorities and multiple sending modalities (for example, best effort versus guaranteed
delivery). \pet{These quality of service features are currently completely unimplemented.} Many
of these additional services would be tedious to provide on a case-by-case basis following the
traditional architecture.

CubedOS also allows any module to potentially send a message to any other module. Thus the
communication paths in the running application are very flexible. In particular, implementing
reusable server modules becomes very straightforward. Each request message contains the module
ID of the sending module, which is simply an index into the mailbox array. The server that
receives the message uses that ID to send the reply to the correct mailbox. There is need for
the server to have prior knowledge of its clients.

Although the CubedOS architecture supports only point-to-point message passing, CubedOS comes
with a library module that supports a publish/subscribe discipline. The module allows multiple
channels to be created to which other modules can subscribe. Publisher modules can then send
messages to one or more channels. Since the messages themselves are unstructured octet streams,
the publish/subscribe module can handle them generically without being modified to account for
new message types.

Every module in a CubedOS system is assigned a module ID number by the application developer. It is our intention to implement a name service that can map module names to
ID numbers dynamically. Roughly, at system initialization time a module registers itself with
the name service and received a dynamically assigned ID number. Other modules can resolve the
name to that ID number by making a query to the name service. This allows third party library
modules the ability to adapt to their environment in a way that would be impossible if they
needed statically assigned addresses. Applications using multiple third party library modules
would have no way to assure conflict free ID assignments using a static method.

We are also defining standard message interfaces to certain services, such as file handling,
that third party modules could implement. This allows modules to use a service without knowing
which specific implementation backs that service.

However, CubedOS's architecture also carries some significant disadvantages as well:

\begin{itemize}

\item
  The common message type also requires that typed information sent from one module to another
  be encoded into a raw octet format when sent, and decoded back into specifically typed data
  when received. The encoding and decoding increases the runtime overhead of message passing and
  reduces type safety. Modules must defend themselves, at runtime, from malformed messages, causing certain errors that were compile-time errors in the
  traditional architecture to now be runtime errors. This is exactly counter to the general
  goals of high integrity system development.

\item In order to return reply messages, the mailboxes must be addressable at runtime using
  module ID numbers. Accessing a statically named mailbox isn't general enough. As a result, the
  precise communication paths used by the system cannot easily be determined statically.

  In particular, since \SPARK\ does not attempt to track information flow through individual
  array elements, it is necessary to manually justify certain \SPARK\ flow messages. \pet{This
    doesn't seem to be true any longer.} Although the architecture of CubedOS ensures that there
  is a one-to-one correspondence between a module and its mailbox. The tools don't know this and
  the spurious flow messages they produce must be suppressed.
\end{itemize}

The details of CubedOS mitigate, to some degree, the problems above. CubedOS does not attempt to provide a one-size-fits-all mailbox array that
will be satisfactory to all applications.

Also every well behaved CubedOS module should contain an |API| package with subprograms for
encoding and decoding messages. This package can be generated by the \texttt{Merc} tool. The
parameters to these subprograms correspond to the parameters of the protected procedures and
entries in the traditional architecture, and provide much of the same type safety. However,
using the API subprograms is not enforced by the compiler. It is technically possible to accidentally
send a message to the wrong mailbox.

So far we have described two extremes: a traditional approach that does not use CubedOS at all,
and an approach that entirely relies on CubedOS. However, hybrid approaches are also possible.
Figure~\ref{fig:hybrid-architecture} shows a combination of several CubedOS mailboxes and a
hand-made, optimized protected object to mediate communication from $T_3$ to $T_4$.

\begin{figure}[tbhp]
  \center
  \scalebox{0.40}{\includegraphics*{Figure-Hybrid.pdf}}
  \caption{Hybrid Architecture}
  \label{fig:hybrid-architecture}
\end{figure}

This provides the best of both worlds. The simplicity and flexibility of CubedOS can be used
where it makes sense to do so, and yet critical communications can still be optimized if the
results of profiling indicate a need. In Figure~\ref{fig:hybrid-architecture} task $T_4$ can't
be reached by CubedOS messages. The hand-made protected object creates a degree of isolation
that can also simplify analysis as compared to a pure CubedOS system. In effect, from CubedOS's
point of view, $T_4$ is an internal task of module \#3 and thus part of module \#3.

It is also possible to instantiate the CubedOS message manager multiple times in the same
application, effectively creating multiple communication domains using separate mailbox arrays.
Figure~\ref{fig:multi-domain} shows an example of where $T_4$ is in a separate domain from the
other modules (because it receives from a mailbox that is separate from the others). In a more
realistic example, the second communication domain would contain more than one module.

\begin{figure}[tbhp]
  \center
  \scalebox{0.40}{\includegraphics*{Figure-MultiDomain.pdf}}
  \caption{Multiple Communication Domains}
  \label{fig:multi-domain}
\end{figure}

This approach allows the CubedOS infrastructure to be used for easy development while still
partitioning the system into semi-independent sections.

Notice in Figure~\ref{fig:multi-domain} tasks $T_3$ and $T_4$ send messages into multiple
domains. This is, of course, sometimes necessary if the domains are going to interact. In such a
scenario each communication domain would regard the other domain as a collection of internal
tasks that is part of the module(s) that send messages into the other domain. We currently have
very little experience with systems designed in this way, but this is an area for future work.
