
\chapter{Architecture}
\label{chapt:architecture}

In this chapter we describe the high level architecture of CubedOS. This includes both the
architecture of the system and the architecture of the development environment.


\section{Overview}
\label{sec:overview}

To understand the context of the CubedOS architecture, it is useful to compare the architecture
of a CubedOS-based application with that of a more traditional application. Since CubedOS abides
by the restrictions of the \textbf{Core.Ravenscar} requirement, we must compare CubedOS with
other Ravenscar-based approaches.

Figure~\ref{fig:traditional-architecture} shows an example application using Ravenscar tasking.
Tasks, which must all be library level infinite loops, are shown in open circles and labeled as
$T_1$ through $T_4$. Tasks communicate with each other via protected objects, shown as solid
circles and labeled as $PO_1$ through $PO_4$.

\begin{figure}[tbhp]
  \center
  \scalebox{0.50}{\includegraphics*{Figure-Traditional.pdf}}
  \caption{Traditional Ravenscar-based Architecture}
  \label{fig:traditional-architecture}
\end{figure}

Arrows from a sending task to a protected object indicate calls to a protected procedure to
install information into the protected object. Arrows from a protected object to a receiving
task indicate calls to an entry in the protected object used to pick up information previously
stored in the object. Entry calls will block if no information is yet available but protected
procedure calls do not block.

Ravenscar requires that protected objects have at most one entry and that at most one task can
be queued on that entry. In CubedOS applications each protected object is serviced by exactly
one task. This ensures that two tasks will never accidentally be queued on the protected
object's entry \emph{provided the mapping from task (actually module, see below) to protected
  object is truly bijective}. In the figure this means only one arrow can emanate from a
protected object, and each task can have only one arrow incident upon it. Note that this
architecture is actually more restrictive than what Ravenscar allows. However, in any case,
multiple arrows can lead to a protected object, since it is permitted in CubedOS for many tasks
to call the same protected procedure or for there to be multiple protected procedures in a given
protected object.

In the example application of Figure~\ref{fig:traditional-architecture}, tasks $T_1$ and $T_3$
call protected procedures in two different protected objects. This presents no problems since
protected procedures never block, allowing a task to call both procedures in a timely manner.
However, task $T_3$ calls two entries, as allowed by Ravenscar but not CubedOS, one in $PO_1$
and another in $PO_2$. Since entry calls can block, this means the task might get suspended on
one of the calls leaving the other protected object without service for an extended time. The
application needs to either be written so that will never happen or be such that it doesn't
matter if it does.

There are several advantages to the traditional organization:

\begin{itemize}
\item The protected objects can be tuned to transmit only the information needed so the overhead
  can be kept minimal.
\item The parameters of the protected procedures and entries specify the precise types of the
  data transferred so compile-time type safety is provided.
\item The communication patterns of the application are known statically, facilitating analysis.
\end{itemize}

However the traditional architecture also includes some disadvantages:

\begin{itemize}
\item The protected objects must all be custom designed and individually implemented, creating a
  burden for the application developer.
\item The communication patterns are relatively inflexible. Changing them requires overhauling
  the application.
\item Third-party library components are awkward to write since they would have to know about
  the protected objects that are available in the application in order to communicate with other
  application-specific components. This is a particular problem for client/server oriented
  systems where a general purpose, reusable server component needs some way to send replies to
  clients for which it has no prior knowledge. \footnote{With access types, the client could
    potentially send a ``return address'' consisting of an access value that points at a
    client-owned protected object implementing an agreed upon synchronized interface. This design
    is ruled out by \SPARK.}
\end{itemize}

The CubedOS version of the sample application has an architecture as shown in
Figure~\ref{fig:cubedos-architecture}. In this case CubedOS provides the communication
infrastructure as an array of general purpose, protected mailbox objects. CubedOS
\textit{modules} communicate by sending messages to the receiver module's mailbox. The messages
are unstructured octet streams, and thus completely generic. Each module has exactly one mailbox
associated with it and contains a single task (the \newterm{server task}) dedicated to servicing
that mailbox alone, creating a one-to-one relationship between modules and mailboxes. A module's
mailbox servicing task extracts messages from the mailbox, decodes the messages, and then acts
on the messages, perhaps sending messages to other modules in the process.

Note that CubedOS modules are allowed to have additional internal tasks, if required, as long as
the usual Ravenscar rules are obeyed. These internal tasks can send messages to other modules,
but they do not attempt to receive messages; that work is reserved for the server task alone.

The communication connections shown in Figure~\ref{fig:cubedos-architecture} are the same as
those shown in Figure~\ref{fig:traditional-architecture} except that the two communication paths
from $T_1$ to $T_3$ are combined into a single path going through one mailbox.

\begin{figure}[tbhp]
  \center
  \scalebox{0.40}{\includegraphics*{Figure-CubedOS.pdf}}
  \caption{CubedOS-based Architecture}
  \label{fig:cubedos-architecture}
\end{figure}

CubedOS relieves the application developer of the problem of creating the communications
infrastructure manually. Adding new message types is simplified with the help of a tool,
\texttt{Merc}, stored in a different repository of the CubeSat Laboratory GitHub. In addition to
providing basic, bounded mailboxes, CubedOS also provides other services such as message
priorities and multiple sending modalities (for example, best effort versus guaranteed
delivery). \pet{These quality of service features are currently completely unimplemented.} Many
of these additional services would be tedious to provide on a case-by-case basis following the
traditional architecture.

CubedOS also allows any module to potentially send a message to any other module. Thus the
communication paths in the running application are very flexible. In particular, implementing
reusable server modules becomes very straightforward. Each request message contains the module
ID of the sending module, which is simply an index into the mailbox array. The server that
receives the message uses that ID to send the reply to the correct mailbox. There is need for
the server to have prior knowledge of its clients.

Although the CubedOS architecture supports only point-to-point message passing, CubedOS comes
with a library module that supports a publish/subscribe discipline. The module allows multiple
channels to be created to which other modules can subscribe. Publisher modules can then send
messages to one or more channels. Since the messages themselves are unstructured octet streams,
the publish/subscribe module can handle them generically without being modified to account for
new message types.

Every CubedOS module has an ID number. These numbers are currently statically assigned by the
application developer with the exception of the CubedOS core modules which have ``well known''
module ID numbers. It is our intention to implement a name service that can map module names to
ID numbers dynamically. Roughly, at system initialization time a module registers itself with
the name service and received a dynamically assigned ID number. Other modules can resolve the
name to that ID number by making a query to the name service. This allows third party library
modules the ability to adapt to their environment in a way that would be impossible if they
needed statically assigned addresses. Applications using multiple third party library modules
would have no way to assure conflict free ID assignments using a static method.

We are also defining standard message interfaces to certain services, such as file handling,
that third party modules could implement. This allows modules to use a service without knowing
which specific implementation backs that service.

However, CubedOS's architecture also carries some significant disadvantages as well:

\begin{itemize}
\item All mailboxes must have the same size since they are stored in an array. This arises
  because the \textbf{Core.SPARK} requirement forbids the use of pointers. Consequently some
  mailboxes will be larger than necessary, wasting space. \pet{\SPARK\ no longer forbids
    pointers outright. Can this requirement be relaxed?}

\item All messages have the same type.\footnote{\textbf{Core.SPARK}'s prohibition on pointers
    prevents more flexible designs. However, a more flexible design might be possible with
    modern \SPARK} This means the size of a message must be large enough to satisfy the needs of
  every module. As a result, in many cases messages will be larger and slower to copy than
  necessary.

  The common message type also requires that typed information sent from one module to another
  be encoded into a raw octet format when sent, and decoded back into specifically typed data
  when received. The encoding and decoding increases the runtime overhead of message passing and
  reduces type safety. Modules must defend themselves, at runtime, from malformed or
  inappropriate messages, causing certain errors that were compile-time errors in the
  traditional architecture to now be runtime errors. This is exactly counter to the general
  goals of high integrity system development.

\item In order to return reply messages, the mailboxes must be addressable at runtime using
  module ID numbers. Accessing a statically named mailbox isn't general enough. As a result, the
  precise communication paths used by the system cannot easily be determined statically.

  In particular, since \SPARK\ does not attempt to track information flow through individual
  array elements, it is necessary to manually justify certain \SPARK\ flow messages. \pet{This
    doesn't seem to be true any longer.} Although the architecture of CubedOS ensures that there
  is a one-to-one correspondence between a module and its mailbox. The tools don't know this and
  the spurious flow messages they produce must be suppressed.
\end{itemize}

The details of CubedOS mitigate, to some degree, the problems above. For example, the mailbox
array is actually instantiated from a generic unit by the application developer. This allows the
developer to tune the sizes of the mailboxes, and the messages they contain, to the
application's needs. CubedOS does not attempt to provide a one-size-fits-all mailbox array that
will be satisfactory to all applications.

Also every well behaved CubedOS module should contain an |API| package with subprograms for
encoding and decoding messages. This package can be generated by the \texttt{Merc} tool. The
parameters to these subprograms correspond to the parameters of the protected procedures and
entries in the traditional architecture, and provide much of the same type safety. However,
using the API subprograms is not enforced by the compiler. It is also possible to accidentally
send a message to the wrong mailbox. Thus modules still need to include runtime error checking
to detect and handle these problems.

So far we have described two extremes: a traditional approach that does not use CubedOS at all,
and an approach that entirely relies on CubedOS. However, hybrid approaches are also possible.
Figure~\ref{fig:hybrid-architecture} shows a combination of several CubedOS mailboxes and a
hand-made, optimized protected object to mediate communication from $T_3$ to $T_4$.

\begin{figure}[tbhp]
  \center
  \scalebox{0.40}{\includegraphics*{Figure-Hybrid.pdf}}
  \caption{Hybrid Architecture}
  \label{fig:hybrid-architecture}
\end{figure}

This provides the best of both worlds. The simplicity and flexibility of CubedOS can be used
where it makes sense to do so, and yet critical communications can still be optimized if the
results of profiling indicate a need. In Figure~\ref{fig:hybrid-architecture} task $T_4$ can't
be reached by CubedOS messages. The hand-made protected object creates a degree of isolation
that can also simplify analysis as compared to a pure CubedOS system. In effect, from CubedOS's
point of view, $T_4$ is an internal task of module \#3 and thus part of module \#3.

It is also possible to instantiate the CubedOS message manager multiple times in the same
application, effectively creating multiple communication domains using separate mailbox arrays.
Figure~\ref{fig:multi-domain} shows an example of where $T_4$ is in a separate domain from the
other modules (because it receives from a mailbox that is separate from the others). In a more
realistic example, the second communication domain would contain more than one module.

\begin{figure}[tbhp]
  \center
  \scalebox{0.40}{\includegraphics*{Figure-MultiDomain.pdf}}
  \caption{Multiple Communication Domains}
  \label{fig:multi-domain}
\end{figure}

This approach allows the CubedOS infrastructure to be used for easy development while still
partitioning the system into semi-independent sections. For example, the sizes of the mailboxes
and of the messages used in each communication domain need not be the same. The parts of the
application that require large messages could be grouped into a domain separate from the parts
that only require small messages.

Notice in Figure~\ref{fig:multi-domain} tasks $T_3$ and $T_4$ send messages into multiple
domains. This is, of course, sometimes necessary if the domains are going to interact. In such a
scenario each communication domain would regard the other domain as a collection of internal
tasks that is part of the module(s) that send messages into the other domain. We currently have
very little experience with systems designed in this way, but this is an area for future work.


% The priority architecture described below has been implemented (although with limited testing)
% in the 'priorities' branch of the GitHub repository. Unfortunately, the prohibition on dynamic
% priorities is enforced by Ravenscar and not SPARK. SPARK requires Ravenscar, and Ravenscar
% does not allow dynamic priorities; any attempt to use them fails to compile. Even upgrading to
% the Jorvik profile, which is also supported by SPARK, does not work around this issue. In
% other words, it's not a matter of merely justifying a SPARK diagnostic as was originally
% assumed. The program won't compile if the priority architecture below is in place.
%
% One potential workaround would be to develop with SPARK and then deploy the program on a full
% runtime system (not Ravenscar) that allows dynamic priorities. This would require switching
% the generic_message_manager package to the one using priorities before compiling the deployed
% system. This isn't a very satisfying approach and may be dangerous. Furthermore, it does
% impose a requirement on the target platform of supporting a full Ada runtime system.
%
% The priorities branch has not been merged into the master branch because the priority system
% either can't be used at all or can only be used under the controversial approach mentioned
% above. However, this matter can be revisited in the future if desired, of course. Note that
% the priorities branch also contains a sample based on the Mars Pathfinder mission which is
% intended to illustrate priority inversion and how the priority architecture fixes the problem.
% If priorities ever become part of the main line of CubedOS development, that sample program
% should be incorporated as well.
%
% 
\section{Message Priorities}
\label{section-message-priorities}

In this section we describe the architecture of message priorities in CubedOS. Priorities are
built on top of the task priority system provided by the Ada runtime environment. CubedOS does
not directly interact with the underlying operating system's notion of priorities. This provides
portability to any system with a suitably capable Ada runtime environment. Specifically, message
priorities are supported on selected bare board systems without an operating system.

\subsection{Justification}

\pet{Why do we need this? Fill out this section when we know!}

\subsection{Architecture}

All modules are assigned a ``base priority'' statically using suitable Ada language pragmas
applied to the module's server task. We define the server task of a module as the task that
reads the module's mailbox and processes the module's messages. We speak of a module's priority
as a shorthand for the ``priority of a module's server task.'' Note that modules are allowed to
have internal tasks in addition to the main task. The priority of those tasks is an internal
matter that does not concern us here.

CubedOS also allows messages to be given priorities that can be distinct from the priorities of
the modules that send and receive them. Message priorities are not a concept known to the Ada
runtime environment; they are purely a construct of CubedOS.

When a module sends a message that message is, by default, given a priority equal to that of the
sending module. However, a module can optionally lower the priority of a message below its own
priority. If a module attempts to raise the priority of a message above its own priority, it is
not an error, either statically or dynamically, but the priority is simply set to the module's
priority instead. \pet{It might be better if this was a statically detected error.}

Messages are removed from a module's mailbox in priority order in $O(\log\,n)$ time (where $n$
is the number of pending messages in the mailbox). \pet{I'm assuming the mailboxes are priority
  heaps} When a module with priority $m$ is processing a message at priority $p > m$, the
module's priority is dynamically increased to $p$ and any messages it sends, including replies
to the original message, are sent by default at priority $p$ (although the module can downgrade
the priority of sent messages, as usual). When a module completes the processing of the high
priority message, just before it fetches the next message from the mailbox, its priority is
returned to its original base priority level.

When a module with priority $m$ is processing a message at priority $p < m$, the module's
priority remains unchanged at its base priority. Thus a high priority module services low
priority messages at its usual (high) priority. In contrast a low priority module services high
priority messages at a temporarily elevated priority. This \newterm{priority inheritance} is
needed so a high priority client isn't slowed down when it requests service from a low priority
module. When executing on behalf of a high priority client, a low priority server inherits the
priority of the client temporarily.

This structure also allows a high priority module to downgrade the priority of outgoing messages
when servicing a low priority client. The idea is that the high priority module might want to
make requests to other modules using the client's priority (which is available to it in the
original request message sent by the client). It is up the application developer to do this if
it is desired. The default behavior is for all sent messages to be given the priority of the
sender.

\subsection{Static Analysis}

Unfortunately \SPARK\ does not currently support the dynamic modification of task priorities.
Thus use of that feature in a CubedOS application must be specifically justified so as to avoid
spurious \SPARK\ diagnostics. However, doing this means that \SPARK\ can no longer be relied
upon to find all potential deadlocks or livelocks. It is therefor necessary to employ an
additional form of analysis that can supplement \SPARK\ and statically show freedom from those
issues in any case. This additional analysis has yet to be defined. It can, however, take
advantage of certain restrictions that still remain in CubedOS tasking despite the addition of
dynamic priorities.

\begin{itemize}
\item Modules only interact via message passing.
\item Messages carry priorities and module priorities are dynamically adjusted as described above.
\item No other dynamic priority adjustments are allowed. \pet{It might be necessary to forbid a
    module from lowering the priority of its outgoing messages\ldots\ perhaps in a first
    implementation anyway} This is statically checked with a supplementary tool. \pet{Using
    libadalang, perhaps?}
\end{itemize}

\pet{I wonder if spin can be used to do what we need. Perhaps a spin module can be automatically
  extracted from the Ada source using a supplementary tool. That same tool could perhaps do the
  additional static checking mentioned above}





\section{Message Encoding}
\label{section-message-encoding}

CubedOS mailboxes store messages as unstructured octet arrays. This allows a general purpose
mailbox package to store and manipulate messages of any type. Unfortunately this also requires
that well structured, well typed message information be encoded to raw octets before being
placed in a mailbox and then decoded after being retrieved from a mailbox.

The CubedOS convention is to use External Data Representation (XDR) encoded messages. XDR is a
well known standard \cite{rfc-4506} that is also simple and has low overhead. We have defined an
extension to XDR that allows \SPARK's constrained scalar subtypes to be represented and that
allows limited contracts to be expressed. The tool \texttt{Merc} compiles a high level
description of a message into message encoding and decoding subprograms. Our tool is written in
Scala and is not verified, but its output is subject to the same \SPARK\ analysis as the rest of
the system. In our case it is easier to prove the output of \texttt{Merc} than it is to prove
the correctness of \texttt{Merc} itself.

The use of \texttt{Merc} mitigates some of CubedOS's disadvantages. The developer need not
manually write the tedious and repetitive encoding and decoding subprograms. Furthermore, those
subprograms have well-typed parameters thus shielding the programer from the inherent lack of
type safety in the mailboxes themselves.

To illustrate CubedOS message handling, consider the following short example of a message
definition file that is acceptable to \texttt{Merc}.

\begin{verbatim}
enum Series_Type { One_Shot, Periodic };

typedef unsigned int Module_ID range 1 .. 16;
typedef unsigned int Series_ID_Type range 1 .. 10000;

message struct {
    Module_ID      Sender;
    Time_Span      Tick_Interval;
    Series_Type    Request_Type;
    Series_ID_Type Series_ID;
} Relative_Request_Message;
\end{verbatim}

This file introduces several types following the usual syntax of XDR interface definitions. The
syntax is extended, however, to allow the programmer to include constrained ranges on the scalar
type definitions in a style that is normal for Ada. The message itself is described as a
structure containing various components in the usual way. The reserved word |message| prefixed
to the structure definition, an extension to XDR, alerts \texttt{Merc} that it needs to generate
encoding and decoding subprograms for that structure. Other structures serving only as message
components can also be defined.

\texttt{Merc} has built-in knowledge of certain Ada private types such as |Time_Span| (from the
|Ada.Real_Time| package). Private types need special handling since their internal structure
can't be accessed directly from the encoding and decoding subprograms. There is currently no
mechanism in \texttt{Merc} to solve this problem in the general case.

Each message type has an ID number that is unique across the application. This is required to
distinguish valid messages from invalid ones. \texttt{Merc} assigns these message type ID
numbers automatically. The encoding subprogram installs the ID number into the message as a
hidden component. The decoding subprogram checks the message type ID number and returns a
failure status if it is invalid. A CubedOS module extracts a message from its mailbox and then
applies the relevant decoder subprograms to that message until one succeeds. If all decoder
subprograms fail the message either has an unexpected type (for example it was accidentally sent
to the wrong mailbox) or the message is malformed in some way. The module must then make a
runtime decision about how to handle that situation.

\texttt{Merc} is a work in progress. We intend to ultimately support as much of the XDR standard
as we can including, for example, variable length arrays and discriminated unions.

