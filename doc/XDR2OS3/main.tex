%%%%%%%%%%%%%%%%%%%%%%%%%%%%%%%%%%%%%%%%%%%%%%%%%%%%%%%%%%%%%%%%%%%%%%%%%%%%
% FILE    : main.tex
% AUTHOR  : (C) Copyright 2018 by Sean Klink & Carl Brandon & Peter Chapin
% SUBJECT : Paper on CubedOS
%
% TODO: Write me! Add more specific items here as they become apparent.
%
% Send comments or bug reports to:
%
%    Carl Brandon
%    Vermont Technical College
%    Randolph Center, VT 05061
%    CBrandon@vtc.vsc.edu
%%%%%%%%%%%%%%%%%%%%%%%%%%%%%%%%%%%%%%%%%%%%%%%%%%%%%%%%%%%%%%%%%%%%%%%%%%%%

%+++++++++++++++++++++++++++++++++
% Preamble and global declarations
%+++++++++++++++++++++++++++++++++
\documentclass{llncs}
\usepackage{listings}
\usepackage{gensymb}
\usepackage{graphicx}
%\usepackage{unicode-math}
\usepackage{url}
\usepackage{hyperref}


\newcommand{\newterm}[1]{\emph{#1}}    % For newly introduced terms.
\newcommand{\code}[1]{\texttt{#1}}     % Used for bits of Ada inline.
\newcommand{\filename}[1]{\texttt{#1}} % For file names.
\newcommand{\SPARK}{\textsc{Spark}}    % For ease of small capping SPARK.
\newcommand{\todo}[1]{\textit{#1}}     % For quick, general to do items.


% Redefine the underscore command so latex will break words at underscores. Found this trick at
% http://mrtextminer.wordpress.com/2009/02/26/break-a-long-word-containing-underscores-in-latex/
%
\let\underscore\_
\newcommand{\breakingunderscore}{\renewcommand{\_}{\underscore\hspace{0pt}}}

% Issue the changed underscore command to the whole document.
\breakingunderscore


% The \note macros are useful for creating easy to see notes.

% Carl Brandon
\long\def\car#1{\marginpar{CAR}{\small \ \ $\langle\langle\langle$\
{#1}\
    $\rangle\rangle\rangle$\ \ }} 

% Peter Chapin
\long\def\pet#1{\marginpar{PET}{\small \ \ $\langle\langle\langle$\
{#1}\
    $\rangle\rangle\rangle$\ \ }} 
	
% General notes from unspecified authors
\long\def\note#1{\marginpar{NTE}{\small \ \ $\langle\langle\langle$\
{#1}\
    $\rangle\rangle\rangle$\ \ }} 


%% The following are settings for the listings package. See the listings package documentation
%% for more information.
%%
%\lstset{language=C,
%        basicstyle=\small,
%        stringstyle=\ttfamily,
%        commentstyle=\ttfamily,
%        xleftmargin=0.25in,
%        showstringspaces=false}

% Define listing parameters for Ada 2012. Base it on Ada 2005.
\lstdefinelanguage{Ada2012}[2005]{Ada}
{
  morekeywords={some},
  sensitive=false,
  breaklines=false,
  showstringspaces=false,
  xleftmargin=0.25in,
  basicstyle=\small\sffamily,
  columns=flexible,
}
\lstset{language=Ada2012, showlines=true}

% Wide equivalent of the listings package lstset. The parameter is the amount to indent the code
% (may be negative).
%
\lstnewenvironment{widelisting}[1]
   {\lstset{language=Ada2012,xleftmargin=#1}}
   {}
   
% Wide listing from a file using the Listings package lstinputlisting. First parameter is
% path name to file (can be relative). Second parameter is the amount to indent the code (may
% be negative).
%
\newcommand{\wideinputlisting}[2]{%
   \lstinputlisting[language=Ada2012,xleftmargin=#2]{#1}}


% An environment for displaying use cases.
%   This environment takes three parameters:
%   \param #1: The name of the use case.
%   \param #2: The actor who participates in the use case.
%   \param #3: The context in which the use case executes.
%
%   The body of the environment is the action associated with the use case.
\newsavebox{\UseCaseName}    % Create some boxes to hold the necessary text.
\newsavebox{\UseCaseActor}   % We need to do this because we can't use the
\newsavebox{\UseCaseContext} % environment parameters in the 'end' definition.
\newsavebox{\UseCaseAction}
\newenvironment{usecase}[3]
 {
  \sbox{\UseCaseName}{\bfseries #1}  % The name is easy.
  \sbox{\UseCaseActor}{#2}           % The actor is easy.
  \begin{lrbox}{\UseCaseContext}     % Format the context in a minipage.
    \begin{minipage}{3.25in}
    #3
    \end{minipage}
  \end{lrbox}
  \begin{lrbox}{\UseCaseAction}      % The environment body becomes the action.
  \begin{minipage}{3.25in}}
 {\end{minipage}
  \end{lrbox}
\begin{center}   % Now spew forth the table using the information collected above.
\begin{tabular}{|l||p{3.5in}|} \hline
\multicolumn{2}{|c|}{\usebox{\UseCaseName}} \\ \hline
Actor   & \usebox{\UseCaseActor}   \\ \hline
Context & \usebox{\UseCaseContext} \\ \hline
Action  & \usebox{\UseCaseAction}  \\ \hline
\end{tabular}
\end{center}}


%++++++++++++++++++++
% The document itself
%++++++++++++++++++++
\begin{document}

\title{Message Encoding for CubedOS Using the XDR2OS3 Code Generator}
\author{Sean Klink\inst{1} \and Peter Chapin\inst{3} \and Carl Brandon\inst{2} }

\institute{Vermont Technical College, Randolph Center VT 05061, USA,\\
  \email{sean.klink@vtc.edu} %,\\
      % WWW home page: \texttt{http://web.vtc.edu/users/xyzzy}
  \and
  \email{carl.brandon@vtc.edu} %,\\
      % WWW home page: \texttt{http://web.vtc.edu/users/xyzzy}
  \and
  Vermont Technical College, Williston VT 05495, USA,\\
  \email{pchapin@vtc.vsc.edu} %,\\
      % WWW home page: \texttt{http://web.vtc.edu/users/pcc09070}
}

\maketitle

\begin{abstract}
  % Summarize the contents of the paper using no more than 400 words.

  We are developing the flight software the Lunar IceCube mission, scheduled to be launched in
  2018. Lunar IceCube will be a 6-U CubeSat orbiting the moon in search of water, ice, and other
  volatiles near the lunar surface. Lunar IceCube will be the first spacecraft to use CubedOS, a
  general purpose, open source application framework for CubeSat flight software written by us.
  CubedOS is written in \SPARK\ 2014 and proved free of certain classes of runtime error. It
  uses a message passing architecture where each concurrently executing module processes
  messages from an associated mail box and sends messages to other mail boxes. CubedOS
  encourages a client/server software architecture where server modules process request messages
  and return reply messages to clients. Message data is encoded using the external data
  representation (XDR) standard. Each module thus provides encoding and decoding subprograms so
  the clients of a module are not required to manage message encodings themselves. These
  subprograms are tedious for module authors to produce manually. In this paper we present
  XDR2OS3, a tool that enables CubedOS module authors to specify message structures using an
  interface language that follows the XDR standard, extended to account for the special
  properties of \SPARK\ (such as constrained subtypes). XDR2OS3 uses the message structures and
  generates a \SPARK\ package containing suitable encoding and decoding subprograms. The output
  of XDR2OS3 can be analyzed by the \SPARK\ tools and proved free of runtime error. While the
  development of XDR2OS3 is a work in progress, our preliminary experience with the tool is
  encouraging. It allows module authors to focus on message types and how they will be used and
  supported rather than on the low level details of message formatting. We also discuss the
  potential for XDR2OS3 to be used in other \SPARK\ applications with an interest in XDR encoded
  data.
  
  \keywords{SPARK, student project, CubeSat}
\end{abstract}


\section{Introduction}

CubedOS is being developed at Vermont Technical College's CubeSat Laboratory with the purpose of
providing a robust software platform for CubeSat missions and of easing the development of
CubeSat flight software. In many respects the goals of CubedOS are similar to those of the Core
Flight Executive (cFE) written by NASA Goddard Space Flight Center \cite{cFE}. However, unlike
cFE, CubedOS is written in \SPARK\ and verified to be free of the possibility of runtime error.
\SPARK\ has also been used to provide some other correctness properties in certain cases. We
compare CubedOS and cFE in more detail in Section~\ref{section-related-work}.

The intent is for CubedOS to be general enough and modular enough for many groups to profitably
employ the system. Since every mission uses different hardware and has different software
requirements, CubedOS is designed as a framework into which \newterm{modules} can be plugged to
implement whatever mission functionality is required. CubedOS provides inter-module
communication and other common services needed by many missions. CubedOS thus serves both as a
kind of operating environment and as a library of useful tools and functions.

Some of the module functionality useful for complex CubeSat missions would include interfaces to
 attitude determination and control systems (ADACS), electrical power systems (EPS), photovoltaic
 panel orientation gimbals, navigation and data radio, data collection instruments, thermal control
 radiators, ion engine with gimbals, and cameras. We also plan on including a specific module for
 spiral thrusting which allows for three axis angular momentum control with a two axis thruster.

It is our intention that all CubedOS modules also be written in \SPARK\ and at least proved free
of runtime error. However, CubedOS allows modules, or parts of modules, to be written in full
Ada or C. This allows CubedOS to take advantage of third party C libraries or to integrate with
an existing C code base.

CubedOS runs on top of the Ada runtime system and thus works with any underlying platform
supported by the available Ada compiler. For example, CubedOS makes use of Ada tasking without
directly invoking the underlying system's support for threads. This simplifies the
implementation of CubedOS while improving its portability. However, CubedOS does require that a
rich Ada runtime system be available for all envisioned targets. Specifically, CubedOS requires
a runtime system that supports the Ravenscar profile.

For resources that are not accessible through the Ada runtime system, CubedOS driver modules can
be written that interact with the underlying operating system or hardware more directly.
Although these modules would not be widely portable, they could, in some cases, be written to
provide a kind of low level abstraction layer (LLAL) with a portable interface. We have not yet
attempted to standardize the LLAL interface. However, we see that as an area for future work.

CubedOS applications are organized as a collection of active and passive modules, where each
active module contains one or more Ada tasks. Passive modules do not contain any tasks but are
used as containers for shared, reusable code. Although CubedOS is written in \SPARK\ there need
not be a one-to-one correspondence between CubedOS modules and \SPARK\ packages. In fact,
modules are routinely written as a collection of Ada packages in a package hierarchy.

Critical to the plug-and-play nature of CubedOS, each active module is self-contained and does
not make direct use of any code in any other active module, although passive modules serving as
library components can be used. All inter-module communication is done through the CubedOS
infrastructure with no direct sharing of data or executable content. In this respect CubedOS
active modules are similar to operating system processes. One consequence of this policy is that
a library used by several modules must be either duplicated in each module, for example as
private child packages, or provided as an independent, passive module. In this respect passive
modules are similar to operating system shared libraries and have similar concerns regarding
task safety and global data management.

In the language of operating systems, CubedOS can be said to have a microkernel architecture
where task and memory management is provided by the Ada runtime system. Both low level
facilities, such as device drivers, and high level facilities, such as communication protocol
handlers, are all implemented as CubedOS modules. All modules are treated equally by CubedOS;
any layered structuring of the modules is imposed by programmer convention.

% TODO: Some of this discussion, along with some diagrams, should perhaps be moved to the
% section describing the architecture of CubedOS. Some orientation to the system is appropriate
% in the introduction... but how much?

CubedOS is currently a work in progress It is our intention to release CubedOS as open source
once it is more mature and refined. We also need to review the code base to verify that it is
free from International Traffic in Arms Regulations (ITAR) restrictions and possibly release both
 ITAR compliant and U.S non ITAR compliant versions. We anticipate this to happen in mid-2018.


\section{Conclusion}
\label{sec:conclusion}

XDR2OS3 is wonderful. Using it will change your life forever.


\bibliographystyle{splncs03}
\bibliography{../../LaTeX/references-General,../../LaTeX/references-SPARK,../../LaTeX/references-CubeSat}

\end{document}
