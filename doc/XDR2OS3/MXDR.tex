
\section{Modified XDR}
\label{sec:mxdr}

Modified eXternal Data Representation (MXDR) is based on the well-known eXternal Data
Representation (XDR) standard \cite{rfc-4506}. Essentially, all syntactic forms in the standard
XDR grammar, except unions, can be found in the MXDR grammar. The main differences between the
MXDR grammar and the XDR grammar are the inclusion of an optional range specifications on types,
the introduction of the message struct as described below, and message invariants.

% Do we want to make these subsections? They might be too short, but it's nice to organize the
% presentation to focus on the additions in MXDR that we outlined in the introduction.

\subsection{Range Constraints}

To specify a range constraint on a type an Ada-like syntax is used as shown below.

\begin{Verbatim}
typedef unsigned int First_Example_Type
    range 1 .. 100;

typedef double Second_Example_Type
    range -3.14159 .. 3.14159;
\end{Verbatim}

It is also permitted to use the following Ada-like syntax for referencing previously declared
bounds:

\begin{Verbatim}
typedef unsigned int Third_Example_Type
    range 2 .. First_Example_Type’Last;
\end{Verbatim}

Standard XDR, which targets a C-like environment, does not support any syntax for specifying
such ranges. However, Ada, and SPARK especially, relies on the generous use of appropriate range
constraints to facilitate the analyses and proofs done by the compiler and tools.

\subsection{Message Structs}

MXDR supports XDR's structure types, which are mapped into Ada records in an obvious way. Such
types can be used just as the primitive types can be used when defining messages. The messages
themselves, however, are defined using a special ``message struct'' that signals to the XDR2OS3
tool that appropriate encoding/decoding subprograms need to be generated.

The contents of a message struct are the parameters of the message. They are mapped to the
parameters of encoding and decoding subprograms as described in
Section~\ref{sec:code-generation}. 

Message structs are sensitive to the direction of message flow. Messages that are intended to be
sent by a module are treated differently than messages that are intended to be received by a
module. This is because the encoding of a message must include the module ID of the sending
module. However, the sending module is different for messages that a given module sends versus
the messages that module receives.

The syntax of message structs therefor include a \texttt{->} symbol for messages that are
inbound to the module defining the message and a \texttt{<-} symbol for messages that are
outbound from that module.

\begin{Verbatim}
message struct -> Example_Request {
    unsigned int        param_1;  // MXDR built-in type
    string              param_2;  // MXDR built-in type
    First_Example_Type  param_3;  // Previously defined
    Second_Example_Type param_4;  // Previously defined
    Third_Example_Type  param_5;  // Previously defined
};

message struct <- Example_Reply {
  int param_1[10];  // XDR arrays fully supported
};
\end{Verbatim}

\subsection{Message Invariants}

In many cases the parameters of a message are required to always have some well defined
interrelationship between their values. For example, a string parameter might always need to be
of some limited size, or a count parameter might always need to be smaller than a bound
parameter. This information needs to be added to the generated code so that SPARK can verify
these constraints are honored by the users of the generated code, and use the constraints when
completing proofs in the generated code. Accordingly MXDR allows \newterm{message invariants} to
be added to message structs as shown in the example below.s:

\begin{Verbatim}
typedef unsigned int Result_Count_Type range 0 .. 512;
typedef unsigned int Result_Bound_Type range 1 .. 512;

message struct <- Example_Reply {
    Result_Count_Type count;
    Result_Bound_Type bound;
} with message_invariant => (count <= bound);
\end{Verbatim}

Currently, only simple relational expressions are supported in message invariants. Support for
additional expression forms, such as quantified expressions for use with arrays, is planned for
the future.

As described in more detail in Section~\ref{sec:code-generation}, message invariants are mapped
to preconditions on encoding functions and post conditions on decoding procedures. This ensures
that the condition expressed by the invariant is try when a message is encoded, and signals the
invariant to the receiving code after the message is decoded. SPARK shows that the generated
code can only produce messages that meet the invariant.

% XDR2OS3 does not necessarily initialize the decided values in accordance with the invariant so
% if a decoding fails, SPARK may not be able to prove the generated code. Solving this problem
% will be difficult.

