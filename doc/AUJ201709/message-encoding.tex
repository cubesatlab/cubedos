
\section{Message Encoding}
\label{section-message-encoding}

\lstMakeShortInline|

CubedOS mailboxes store messages as unstructured octet arrays. This allows a general purpose
mailbox package to store and manipulate messages of any type. Unfortunately this also requires
that well structured, well typed message information be encoded to raw octets before being
placed in a mailbox and then decoded after being retrieved from a mailbox.

The CubedOS convention is to use External Data Representation (XDR) encoded messages. XDR is a
well known standard \cite{rfc-4506} that is also simple and has low overhead. We have defined an
extension to XDR that allows \SPARK's constrained scalar subtypes to be represented. We are
currently working on a tool, XDR2OS3, that will compile a high level description of a message
into message encoding and decoding subprograms. Our tool is written in Scala and is not
verified, but its output is subject to the same \SPARK\ analysis as the rest of the application.
It is easier to prove the output of XDR2OS3 than it is to prove the correctness of XDR2OS3
itself.

The use of XDR2OS3 mitigates some of CubedOS's disadvantages. The developer need not manually
write the tedious and repetitive encoding and decoding subprograms. Furthermore, those
subprograms have well-typed parameters thus shielding the application programer from the
inherent lack of type safety in the mailboxes themselves.

The use of XDR encoding may seem like an odd choice since XDR was originally defined for use in
networking applications where data must be sent between potentially heterogenous systems. Since
we envision current CubedOS applications being written entirely in \SPARK\ and executing in a
single process, XDR seems like a needless complication. However, as described in
Section~\ref{section-architecture}, we anticipate extending CubedOS to work in exactly the kind
of potentially heterogenous environment XDR was developed to support. Thus we aim to provide a
single standard for message encoding that will work in both the near and long term.

To illustrate CubedOS message handling, consider the following short example of a message
definition file that is acceptable to XDR2OS3.

\begin{verbatim}
enum Series_Type { One_Shot, Periodic };

typedef unsigned int Module_ID range 1 .. 16;
typedef unsigned int Series_ID_Type range 1 .. 10000;

message struct {
    Module_ID      Sender;
    Time_Span      Tick_Interval;
    Series_Type    Request_Type;
    Series_ID_Type Series_ID;
} Relative_Request_Message;
\end{verbatim}

This file introduces several types following the usual syntax of XDR interface definitions. The
syntax is extended, however, to allow the programmer to include constrained ranges on the scalar
type definitions in a style that is normal for Ada. The message itself is described as a
structure containing various components in the usual way. The reserved word |message| prefixed
to the structure definition, another XDR extension, alerts XDR2OS3 that it needs to generate
encoding and decoding subprograms for that structure. Other structures serving only as message
components (parameters) can also be defined.

XDR2OS3 has built-in knowledge of certain Ada private types such as |Time_Span| (from the
|Ada.Real_Time| package). Private types need special handling since their internal structure
can't be accessed directly from the encoding and decoding subprograms. There is currently no
mechanism in XDR2OS3 to solve this problem in the general case.

Each message type has an ID number that is scoped by the module that defines the message. Upon
receiving a message, the first step in message handling is to verify that the module ID of the
receiver in the message header agrees with the ID of the module that is processing that message.
This ensures that the message was actually sent to the intended module. Once that is done, the
module is free to interpret the message ID value locally. Message ID values are never directly
visible to module clients since the client calls a named encoding procedure to build each
message. Thus the value and meaning of the message IDs defined by a module is entirely an
implementation matter for the module. XDR2OS3 defines an enumeration type that specifies a
module's messages as easy to read enumerators. It then uses the position value of a message
enumerator as the message ID value.

XDR2OS3 is a work in progress. We intend to ultimately support as much of the XDR standard as we
can including, for example, variable length arrays and discriminated unions. The development of
XDR2OS3 is guided by our immediate needs with our currently envisioned missions, but we intend to extend and
generalize the functionality of XDR2OS3 as the tool matures.

There are other possible encoding and decoding schemes that could have been used. For example,
ASN.1 \cite{asn.1} is another standard with approximately similar goals as XDR. However, ASN.1
is much more complicated and entails more overhead both in space and time. ASN.1 includes type
information in the encoded message itself, however, which may have advantages for error
detection and handling. Since the application developer invokes tool-generated encoding and
decoding procedures, and does not directly deal with message encoding, it would be possible to
switch the message encoding method without significantly impacting applications. A future
version of XDR2OS3 could potentially provide an ASN.1 mode (as one possiblity), perhaps for
reasons of error handling or interoperability with legacy systems. This is also an area for
future work.

\lstDeleteShortInline|
