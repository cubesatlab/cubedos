
\section{Message Encoding}
\label{section-message-encoding}

CubedOS mailboxes store messages as unstructured octet arrays. This allows a general purpose
mailbox package to store and manipulate messages of any type. Unfortunately this also requires
that well structured, well typed message information be encoded to raw octets before being
placed in a mailbox and then decoded after being retrieved from a mailbox.

The CubedOS convention is to use External Data Representation (XDR) encoded messages. XDR is a
well known standard \cite{rfc-4506} that is also simple and has low overhead. We have defined an
extension to XDR that allows \SPARK's constrained scalar subtypes to be represented and that
allows limited contracts to be expressed. The tool \texttt{Merc} compiles a high level
description of a message into message encoding and decoding subprograms. Our tool is written in
Scala and is not verified, but its output is subject to the same \SPARK\ analysis as the rest of
the system. In our case it is easier to prove the output of \texttt{Merc} than it is to prove
the correctness of \texttt{Merc} itself.

The use of \texttt{Merc} mitigates some of CubedOS's disadvantages. The developer need not
manually write the tedious and repetitive encoding and decoding subprograms. Furthermore, those
subprograms have well-typed parameters thus shielding the programer from the inherent lack of
type safety in the mailboxes themselves.

To illustrate CubedOS message handling, consider the following short example of a message
definition file that is acceptable to \texttt{Merc}.

\begin{verbatim}
enum Series_Type { One_Shot, Periodic };

typedef unsigned int Module_ID range 1 .. 16;
typedef unsigned int Series_ID_Type range 1 .. 10000;

message struct {
    Module_ID      Sender;
    Time_Span      Tick_Interval;
    Series_Type    Request_Type;
    Series_ID_Type Series_ID;
} Relative_Request_Message;
\end{verbatim}

This file introduces several types following the usual syntax of XDR interface definitions. The
syntax is extended, however, to allow the programmer to include constrained ranges on the scalar
type definitions in a style that is normal for Ada. The message itself is described as a
structure containing various components in the usual way. The reserved word |message| prefixed
to the structure definition, an extension to XDR, alerts \texttt{Merc} that it needs to generate
encoding and decoding subprograms for that structure. Other structures serving only as message
components can also be defined.

\texttt{Merc} has built-in knowledge of certain Ada private types such as |Time_Span| (from the
|Ada.Real_Time| package). Private types need special handling since their internal structure
can't be accessed directly from the encoding and decoding subprograms. There is currently no
mechanism in \texttt{Merc} to solve this problem in the general case.

Each message type has a message ID that is unique across the application. This is required to
distinguish valid messages from invalid ones. \texttt{Merc} assigns these message type IDs automatically. The encoding subprogram installs the ID into the message as a
hidden component. The decoding subprogram checks the message type ID and returns a
failure status if it is invalid. Modules declare statically what message types it is prepared to receive and the message system discards unacceptable messages before they reach their destination. A CubedOS module reads a message from its mailbox, reads its message type and then
disbatches it to the relevant decoder subprograms.

\texttt{Merc} is a work in progress. We intend to ultimately support as much of the XDR standard
as we can including, for example, variable length arrays and discriminated unions.
