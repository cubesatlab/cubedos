\chapter{Introduction}
\label{chapt:introduction}

This document describes the design, implementation, and usage of the CubedOS application
framework. CubedOS was developed in Vermont Technical College's CubeSat Laboratory with the
purpose of providing a robust software platform for CubeSat missions and of easing the
development of CubeSat flight software. In many respects the goals and architecture of CubedOS
are similar to those of the Core Flight System (CFS) written by NASA Goddard Space Flight Center
\cite{cFE}. However, unlike CFS, CubedOS is written in \SPARK\ with critical sections verified
to be free of the possibility of runtime error. We compare CubedOS and CFS in more detail in
Section~\ref{chapt:related-work}.

Although CubedOS was developed to support the CubeSat missions at Vermont Technical College, the
intent is for CubedOS to be general enough and modular enough for other groups to profitably
employ the system. Since every mission uses different hardware and has different software
requirements, CubedOS is designed as a framework into which \newterm{modules} can be plugged to
implement whatever mission functionality is required. CubedOS provides inter-module
communication and other common services required by many missions. CubedOS thus serves both as a
kind of operating environment and as a library of useful tools and functions.

It is our intention that all CubedOS modules also be written in \SPARK\ and at a minimum proved
free of runtime error. However, CubedOS also allows modules, or parts of modules, to be written
in full Ada or C if desired. This allows CubedOS to take advantage of third party C libraries or
to integrate with an existing C code base. It is a work in progress to create a bridge between
CubedOS and CFS so that current CFS users can experiment with CubedOS and eventually migrate
mission critical components to CubedOS.

CubedOS can run either directly on top of the hardware, or on top of an operating system such as
Linux or VxWorks, all with the assistance of the Ada runtime system. In particular, CubedOS
makes use of Ada tasking without directly invoking the underlying system's support for threads.
This simplifies the implementation of CubedOS while improving its portability. However, it does
require that an Ada runtime system be available for all envisioned targets. In effect, the Ada
runtime system is CubedOS's operating system abstraction layer (to borrow a term from the CFS
community).

For resources that are not accessible through the Ada runtime system, CubedOS driver modules can
be written that interact with the underlying operating system or hardware more directly.
Although these modules would not be widely portable, they could, in some cases, be written to
provide a kind of low level abstraction layer (LLAL) with a portable interface. We have not yet
attempted to standardize an LLAL interface, although this is an area for future work.

The architecture of CubedOS is a collection of modules, each containing one or more Ada tasks.
Although CubedOS is written in \SPARK\ there need not be a one-to-one correspondence between
CubedOS modules and \SPARK\ packages. In fact, modules are routinely written as a collection of
\SPARK\ packages in a package hierarchy. \pet{Provide a forward reference.}

Critical to the plug-and-play nature of CubedOS, each module is self-contained and does not make
direct use of any code that is private to another module. However, CubedOS does allow shared
library packages to be written for holding utility subprograms of various kinds. The CubedOS
distribution includes a collection of such library packages that are used by CubedOS core
modules, but that can also be used by application modules. \pet{It is necessary for shared
  library packages to have certain properties to ensure ``good behavior.'' Does pragma Pure say
  what is needed?}

All inter-module communication is done through the CubedOS infrastructure with no direct sharing
of data or executable content. In this respect CubedOS modules are similar to processes in a
conventional operating system. One consequence of this policy is that a library used by several
modules must be task-safe. In this respect library packages resemble shared libraries in a
conventional operating system and have similar concerns regarding library-wide global data
management.

In the language of operating systems, CubedOS can be said to have a microkernel architecture
where task management is provided by the Ada runtime system. Both low level facilities, such as
device drivers, and high level facilities, such as communication protocol handlers or navigation
algorithms, are all implemented as CubedOS modules. All modules are treated equally by CubedOS;
any layered structuring of the modules is imposed by programmer convention.

We are working on extending the architecture to allow communication to occur between modules in
different operating system processes or even on different hosts. This extension will allow
CubedOS applications to be distributed across multiple spacecraft while still operating as a
single, integrated application. \pet{Provide a forward reference.}

% TODO: Some of this discussion, along with some diagrams, should perhaps be moved to the
% section describing the architecture of CubedOS. Some orientation to the system is appropriate
% in the introduction... but how much?

In addition to inter-module communication, CubedOS provides several general purpose, core
services. In some cases different implementations of a service can be exchanged for that
provided by CubedOS (e. g., the File Server interface might be implemented in various ways
depending on the needs of each mission). Having a standard interface allows other components of
CubedOS to be programmed against that interface without concern about the underlying
implementation used. The following features are either available or are planned to be
implemented:

\begin{itemize}
\item An asynchronous message passing infrastructure with, eventually, the ability to pass
  messages across operating system processes or even across different hosts. This, together with
  the underlying Ada runtime system, constitutes the kernel of CubedOS.
\item A runtime library of useful packages, all verified with \SPARK.
\item Timing services.
\item File services. This provides access to non-volatile storage via a general purpose file
  system abstraction that might be built on different underlying implementations.
\item Event services.
\item Command scripting services.
\item Publish/Subscribe services. This provides the well known publish/subscribe communication
  model on top of CubedOS's native point-to-point message passing system.
\item Log and telemetry gathering services.
\item Name resolution services. This provides a mapping between abstract module names and their
  module ID values. It also provides a registration service for dynamic module ID assignments.
\end{itemize}

A CubedOS system also requires drivers for the various hardware components in the system.
Although the specific drivers required will vary from mission to mission, CubedOS does provide a
general \newterm{driver model} that allows components to communicate with drivers fairly
generically. In a typical system there will be low level drivers for accessing hardware buses
as well as higher level drivers for sending/receiving commands from subsystems such as the
radio, the power system, the camera, etc. The low level drivers constitute the bulk of the
CubedOS LLAL.

CubedOS is currently a work in progress. It is our intention to release CubedOS as open source
once it has been further refined.
